\documentclass[a4paper]{article}
\usepackage{Sweave}
\begin{document}

\title{}
\author{}

\maketitle


\begin{Schunk}
\begin{Sinput}
> require(reshape)
> cleanAndNormalize <- function(dge, clean=TRUE, norm=FALSE) {
+     # dge: an object of DGEList, e.g., dat.dge
+     # clean: boolean; TRUE: keep only with cpm > 1 in more than 10 samples
+     # norm: boolean; TRUE: normalize
+     if (clean & norm) {
+         cpm <- cpm(dge)
+         # THRESHOLD hard coded
+         keep <- rowSums(cpm > 1) > 10
+         
+         # normalize first
+         dge <- calcNormFactors(dge, method='TMM')
+         dge <- dge[keep,, keep.lib.sizes=FALSE]
+ 
+         return(dge)
+     }
+ 
+     if (clean) {
+         cpm <- cpm(dge)
+         keep <- rowSums(cpm > 1) > 10
+         
+         return(dge[keep,, keep.lib.sizes=FALSE])
+     }
+     
+     if (norm) {
+         dge <- calcNormFactors(dge, method='TMM')
+         
+         return(dge)
+     }
+         
+     return(dge)
+ }
> genSubsetAbundMat_RNAseq <- function(vectacc, dgeobj, logtr) {
+     # generates data.frame count matrix given a listing of accession
+     # returns a list of the data frames
+     # vectofacc: listing of accessions
+     # dgeobj: an object of class DGEList (edgeR library); data neither normalized
+     # nor cleaned
+     cnobj <- cleanAndNormalize(dgeobj, clean=TRUE, norm=TRUE)
+         
+     df.lst <- list()
+     spls <- sapply(rownames(cnobj$samples), function(x) unlist(strsplit(x, split='_'))[1])
+     for (spl in as.character(spls)) {
+         x <- cnobj[cnobj$genes$Acc %in% vectacc,grep(spl,rownames(cnobj$samples))]
+ 
+         rownames(x$counts) <- x$genes$Symbol
+         #xcnt <- x$counts
+         
+         if (logtr) {
+             xcpm <- cpm(x, log=TRUE)
+         } else {
+             xcpm <- cpm(x)
+         }
+         
+         spl <- tolower(spl)
+         df.lst[[spl]] <- as.data.frame(xcpm)
+     }
+     
+     return(df.lst)
+ }
> lattice_barchart <- function(df) {
+     xdf <- namerows(df, col.name='Protein')
+     xdf <- melt(xdf, id.var='Protein')
+     v <- stringr::str_split_fixed(xdf$variable, '_', 2)
+     xdf <- data.frame(xdf, v)
+     res <- barchart(value ~ X2|Protein, groups=X1, data=xdf, scales=list(x=list(rot=90)), par.settings=list(superpose.polygon=list(col='lightgreen')))
+     #barchart(value ~ X2|Protein+X1, data=ma, scales=list(x=list(rot=90)), par.settings=list(superpose.polygon=list(col='blue')))
+     return(res)
+ }
> gplot_barplot_RNAseq <- function(df, type='Protein') {
+     ## parameters:
+     # df: data.frame sample data
+     # type: either 'Protein' or 'Transcript'
+     v <- stringr::str_split_fixed(colnames(df), '_', 2) # this line and next moved D21217
+     title <- paste(unique(v[,1]), type, sep=' ')
+     df <- modTMTsixplexLabelNames(df) # added D21217
+     
+     xdf <- namerows(df, col.name='Protein')
+     xdf <- melt(xdf, id.var = 'Protein')
+     colnames(xdf) <- c('Protein', 'Samples', 'Abundance')
+         
+     pg <- ggplot(xdf, aes(Samples, Abundance)) + geom_bar(stat='identity', fill='lightblue', alpha=1)
+     pg <- pg + labs(x='Development Stage', size=10)
+     pg <- pg + facet_grid(. ~ Protein) + ggtitle(title)
+     pg <- pg + theme(axis.text.x = element_text(angle=90, hjust=1), plot.title=element_text(color='blue', hjust=0.5))
+     
+     return(pg) # for multiplot
+ }
> multiBarPlot_RNAseq <- function(pdfl, rdfl) {
+     # pdfl, a list of protein data frames
+     #rdfl, a list of RNAseq data frames
+     pl <- NULL
+     rspls <- names(rdfl)
+     for (nm in names(pdfl)) {
+         sym <- unlist(mget(rownames(pdfl[[nm]]), eacc2sym, ifnotfound=unlist(mget(rownames(pdfl[[nm]]), eBBacc2sym, ifnotfound=rownames(pdfl[[nm]])))))
+         rownames(pdfl[[nm]]) <- sym
+ 
+         pl[[paste(nm, 'prot')]] <- gplot_barplot_RNAseq(pdfl[[nm]], 'Protein')
+         pl[[paste(nm, 'rna')]] <- gplot_barplot_RNAseq(rdfl[[nm]], 'Transcript')
+         rspls <- rspls[!rspls %in% nm]
+     }
+ 
+     for (rnm in rspls) {
+         pl[[rnm]] <- gplot_barplot_RNAseq(rdfl[[rnm]])
+     }
+ 
+     return(pl)
+ }
> ## ==== This function taken from the web ====
> # Multiple plot function
> #
> # ggplot objects can be passed in ..., or to plotlist (as a list of ggplot objects)
> # - cols:   Number of columns in layout
> # - layout: A matrix specifying the layout. If present, 'cols' is ignored.
> #
> # If the layout is something like matrix(c(1,2,3,3), nrow=2, byrow=TRUE),
> # then plot 1 will go in the upper left, 2 will go in the upper right, and
> # 3 will go all the way across the bottom.
> #
> multiplot <- function(..., plotlist=NULL, file, cols=1, layout=NULL) {
+     library(grid)
+ 
+     # Make a list from the ... arguments and plotlist
+     plots <- c(list(...), plotlist)
+ 
+     numPlots = length(plots)
+ 
+     # If layout is NULL, then use 'cols' to determine layout
+     if (is.null(layout)) {
+         # Make the panel
+         # ncol: Number of columns of plots
+         # nrow: Number of rows needed, calculated from # of cols
+         layout <- matrix(seq(1, cols * ceiling(numPlots/cols)),
+                          ncol = cols, nrow = ceiling(numPlots/cols))
+     }
+ 
+     if (numPlots==1) {
+         print(plots[[1]])
+ 
+     } else {
+         # Set up the page
+         grid.newpage()
+         pushViewport(viewport(layout = grid.layout(nrow(layout), ncol(layout))))
+ 
+         # Make each plot, in the correct location
+         for (i in 1:numPlots) {
+         # Get the i,j matrix positions of the regions that contain this subplot
+             matchidx <- as.data.frame(which(layout == i, arr.ind = TRUE))
+ 
+             print(plots[[i]], vp = viewport(layout.pos.row = matchidx$row,
+                                             layout.pos.col = matchidx$col))
+         }
+     }
+ }
> 
> 
\end{Sinput}
\end{Schunk}
\end{document}
